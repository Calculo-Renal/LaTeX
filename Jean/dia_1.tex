\documentclass{article}

\usepackage[brazil]{babel}
\usepackage[T1]{fontenc}
\usepackage[utf8]{inputenc}

\usepackage{listings}

\setlength{\parskip}{0.6em}
\lstset{
    basicstyle=\ttfamily,
    xleftmargin=2em
}

\author{Jean Carlo}
\title{Primeiro dia estudando LaTeX}

\begin{document}

\maketitle
\tableofcontents

\section{O que é o LaTeX?}

O \LaTeX{} é uma extensão do \TeX{} que permite que você digite de forma mais rápida e concisa operações presentes no \TeX{} através do uso de \textbf{macros}.

Além disso, o \LaTeX{} possui um vasto conjunto de operações matemáticas já codificadas em macros, facilitando a digitação de artigos científicos e livros de caráter matemático, por exemplo. 

\section{Como digitar em LaTeX}

\subsection{A classe do documento}

Em \LaTeX{} sempre iniciamos o documento declarando a sua \textbf{classe}, ou seja, o seu caráter. Isso possibilita a padronização de layouts e dá uma pista do objetivo do arquivo. Sendo assim, iniciamos com

\begin{lstlisting}
    \documentclass{<classe>}
\end{lstlisting}

Onde \texttt{<classe>} é a classe a qual o documento pertence. A mais comum delas é \texttt{article}, que define um artigo, normalmente curto (assim como esse!!!).

\subsection{Perambulando por pacotes? Use o preâmbulo!}

Agora vem onde definimos as coisas mais importantes: pacotes, macros que definem informação de cabeçalho, configurações acerca de tipografia\ldots{} Toda essa seção que contém essas coisas e fica entre o comando 

\begin{lstlisting}
    \documentclass{<classe>}
\end{lstlisting}

e do comando 

\begin{lstlisting}
    \begin{document}
\end{lstlisting}

se chama \textbf{preâmbulo}.

\subsubsection{Pacotes}

\textbf{Pacotes} são extensões do \TeX{} que são proporcionadas pelo \LaTeX{}. Eles são definidos através do comando

\begin{lstlisting}
    \usepackage{<nome_do_pacote>}
\end{lstlisting}

Onde \texttt{\textless{}nome\_do\_pacote\textgreater{}} se refere literalmente ao identificador do pacote.

Um pacote que usei inclusive para digitar esse documento é o \texttt{listings}, ele permite usar o comando \texttt{\textbackslash begin} com o parâmento obrigatório \texttt{\{lstlisting\}} para fazer blocos de código como esse:

\begin{lstlisting}
    Olha só, estou muito organizado!
\end{lstlisting}

Outros pacotes que usei também foram:

\begin{description}
    \item[babel] Usei o comando \texttt{\textbackslash usepackage[brazil]\{babel\}} para deixar o \LaTeX{} em português em relação aos comando e macros automáticas. 
    \item[fontenc] Usei o comando \texttt{\textbackslash usepackage[T1]\{fontenc\}} para definir a codificação de fonte para T1.
    \item[inputenc] Usei o comando \texttt{\textbackslash usepackage[utf8]\{inputenc\}} para definir a codificação do input para UTF-8.
\end{description}

\subsection{Outras macros}

Como não sei bem como funcionam e apenas usei por conveniência, irei apenas dar uma breve lista de algumas macros que usei no preâmbulo. São elas:



\section{Alguns dos erros que encontrei...}

\subsection{Avisos (Warnings)}



\subsection{Erros (Errors)}

\end{document}
