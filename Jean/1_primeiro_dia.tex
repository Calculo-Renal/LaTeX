\documentclass{article}

\usepackage[brazil]{babel}
\usepackage[T1]{fontenc}
\usepackage[utf8]{inputenc}

\usepackage[hidelinks]{hyperref}
\usepackage{listings}

\setlength{\parskip}{0.6em}
\lstset{
    basicstyle=\ttfamily,
    xleftmargin=2em
}

\author{Jean Carlo}
\title{Primeiro dia estudando LaTeX}

\begin{document}

\maketitle
\tableofcontents

\section{O que é o LaTeX?}

O \LaTeX{} é uma extensão do \TeX{} que permite que você digite de forma mais rápida e concisa operações presentes no \TeX{} através do uso de \textbf{macros}.

Além disso, o \LaTeX{} possui um vasto conjunto de operações matemáticas já codificadas em macros, facilitando a digitação de artigos científicos e livros de caráter matemático, por exemplo. 

\section{Como digitar em LaTeX}

\subsection{A classe do documento}

Em \LaTeX{} sempre iniciamos o documento declarando a sua \textbf{classe}, ou seja, o seu caráter. Isso possibilita a padronização de layouts e dá uma pista do objetivo do arquivo. Sendo assim, iniciamos com

\begin{lstlisting}
    \documentclass{<classe>}
\end{lstlisting}

Onde \texttt{<classe>} é a classe a qual o documento pertence. A mais comum delas é \texttt{article}, que define um artigo, normalmente curto (assim como esse!!!).

\subsection{Perambulando por pacotes? Use o preâmbulo!}

Agora vem onde definimos as coisas mais importantes: pacotes, macros que definem informação de cabeçalho, configurações acerca de tipografia\ldots{} Toda essa seção que contém essas coisas e fica entre o comando 

\begin{lstlisting}
    \documentclass{<classe>}
\end{lstlisting}

e do comando 

\begin{lstlisting}
    \begin{document}
\end{lstlisting}

se chama \textbf{preâmbulo}.

\subsubsection{Pacotes}

\textbf{Pacotes} são extensões do \TeX{} que são proporcionadas pelo \LaTeX{}. Eles são definidos através do comando

\begin{lstlisting}
    \usepackage{<nome_do_pacote>}
\end{lstlisting}

Onde \texttt{\textless{}nome\_do\_pacote\textgreater{}} se refere literalmente ao identificador do pacote.

Um pacote que usei inclusive para digitar esse documento é o \texttt{listings}, ele permite usar o comando \texttt{\textbackslash{}begin} com o parâmento obrigatório \texttt{\{lstlisting\}} para fazer blocos de código como esse:

\begin{lstlisting}
    Olha só, estou muito organizado!
\end{lstlisting}

Outros pacotes que usei também foram:

\begin{description}
    \item[babel] Define o idioma.
    \item[fontenc] Configura a codificação de fontes.
    \item[inputenc] Configura a codificação de caracteres.
    \item[listings] Para blocos de código.
    \item[hyperref] Para sumário e referências clicáveis.
\end{description}

\subsection{Outras macros}

Como não sei bem como funcionam e apenas usei por conveniência, irei apenas dar uma breve lista de algumas macros que usei no preâmbulo. São elas:

\begin{description}
    \item[setlength] Ajusta medidas como o espaçamento de parágrafos com \texttt{\textbackslash{}parskip}.
    \item[author] Define o autor.
    \item[title] Define o título.
\end{description}

\section{Alguns dos erros que encontrei...}

\subsection{Avisos (Warnings)}

\subsubsection{No pages of output.}

Aqui está o aviso completo:

\begin{lstlisting}
    No pages of output.
\end{lstlisting}

Concluí que ele é gerado quando o \LaTeX{} compila corretamente. Todavia, nada é gerado.

\subsubsection{Overfull \textbackslash hbox (\emph{size} too wide)
}

Aqui está o exemplo de um aviso completo:

\begin{lstlisting}
    Overfull \hbox (78.81297pt too wide)
\end{lstlisting}

Concluí que ele é gerado quando o \LaTeX{} gera um conteúdo que não pôde ser quebrado corretamente horizontalmente, quebrando a \texttt{hbox}.

\subsection{Erros (Errors)}

\subsubsection{Failure to make 'teste.pdf'.}

Aqui está o erro completo:

\begin{lstlisting}
    Failure to make 'teste.pdf'
    pdflatex: failed to create output file
\end{lstlisting}

Concluí que ele é gerado principalmente quando há erros na compilação. Todavia, também pode vir da própria instalação, configuração ou implementação do \texttt{pdflatex}.

\subsubsection{Missing \$ inserted.}

Aqui está o erro completo:

\begin{lstlisting}
    Missing $ inserted.
    <inserted text> LaTeX
\end{lstlisting}

Concluí que ele ocorre quando o \LaTeX{} entra em modo matemático acidentalmente ou senão quando está faltando o \$ para finalizar o modo matemático.

\subsubsection{Extra \}, or forgotten \$.}

Aqui está o erro completo:

\begin{lstlisting}
    Extra }, or forgotten $.
    <recently read> \egroup LaTeX
\end{lstlisting}

No meu caso, era um efeito cascata devido ao erro anterior. O meu \LaTeX{} entrou em modo matemático devido à falta de chaves, que fazia uma má compilação ocorrer.

\end{document}
